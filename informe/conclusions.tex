\section{Conclusions}
Durant la realització d'aquest projecte s'han apres (i sobretot consolidat) molts conceptes de Machine Learning (Aprenentatge Automàtic) i les seves aplicacions en el món real; en aquest cas, en l'àmbit de la salut.

Inicialment, ja s'ha vist que el dataset era complex i requeria un bon tractament per tal de els models fossin bons. Pel que fa al tractament que s'ha fet en aquest projecte, es podrien haver provat més mètodes d'imputació (o haver provat diferents paràmetres) per tal d'obtenir millors resultats. En un principi no semblava que la eliminació de files, outliers o els mètodes de imputació fossin tant determinants pels models, però s'ha acabat veient que canviaven molt el seu comportament.

El rendiment dels models generats semblava bo en un inici, però posteriorment s'ha vist (almenys pel model escollit), que els experiments s'haurien d'haver fet amb més llavors per tal de trobar models que realment generalitzessin millor i no destaquessin només en una llavor. No obstant, com ja s'ha dit al llarg del treball, l'objectiu d'aquesta pràctica ha esta provar diferents mètodes, models... i extreure'n un aprenentatge, de manera que es valora molt veure que certes coses podrien ser millorades, especialment si es tractés d'un model que realment s'hagués d'aplicar al món real. 

Addicionalment, ha quedat en evidència que els petits canvis en el dataset poden influir molt en els resultats dels models, i els propis resultats dels models poden ser molt enganyosos. En futurs projectes, s'haurà d'anar més en compte amb les mètriques i plantejar la creació de models des d'un altre punt de vista. És a dir, en aquest projecte s'han determinat molts paràmetres en funció de les puntuacions, i aquestes puntuacions han resultat ser enganyoses en certs casos, especialment quan es treballa amb una sola llavor. Seria una bona opció haver tingut més en compte altre mètriques més robustes i haver determinat els paràmetres també amb altres criteris.

Pel que fa a les prediccions i les conclusions de l'estudi, es pot dir que les diferències entre pacients amb classe `Alive' i `Dead' semblaven bastant clares. No obstant, els pacients amb `LiverTransplant' poden ser un problema pels models per la seva poca presència en el datset i perquè es desconeix els motius que porten a que un pacient pugui rebre un transplantament de fetge. Per altra banda, és curiós que les dades de l'estudi mostren que la variable \textit{Drug} (que indica si s'ha administrat el tractament o un placebo) és de les que menys influeix en la resposta, de manera que el tractament que es subministrava en aquest estudi no era útil. Totes aquestes coses remarquen la importància d'aquest estudi a nivell d'anàlisi, i no tant de rendiment dels models.

En referència al contingut que es requeria en aquesta pràctica, es considera que generalment s'ha assolit amb èxit. No obstant, com ja s'ha mencionat, hi ha certs aspectes que es podrien millorar (i que es milloraran en futurs projectes) per tal de proporcionar un anàlisi més complet i robust. Pel que fa als punts de bonus, no ha donat temps de poder realitzar-los tots, però s'ha inclòs una petita part del bonus 2, que si s'hagués disposat de més temps hagués pogut ser molt interessant.

Finalment, es pot concloure que l'aprenentatge adquirit en aquesta pràctica, tot i ser el primer contacte que es té amb l'aprenentatge automàtic, és immensament valuós per el grau que s'està cursant. Realment la intel·ligència artificial pot tenir grans usos en molts àmbits del món real, i en aquest projecte s'ha pogut apreciar la gran quantitat d'informació útil que es pot extreure d'una base de dades (tot i que aquesta tingui numerosos desperfectes) i aquest informació es pot utilitzar per generar models capaços de realitzar prediccions que poden ajudar a prendre moltes decisions en el món real.

