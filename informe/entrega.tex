\section{Documents i estructura general de l'entrega}
En l'entrega d'aquesta pràctica s'hi inclou aquest informe juntament amb un arxiu comprimit \texttt{.zip}. 

A dins d'aquest arxiu comprimit es poden trobar dues carpetes: assets i src. En el directori assets es troben altres directoris i un arxiu per importar les llibreries necessaries pel codi. En aquests directoris es guarden els arxius \texttt{.csv} amb les dades que es guarden (en cas d'especificar-ho així en els paràmetres) en algunes execucions, així com les imatges generades en algunes altres execucions (en la entrega no venen totes les imatges generades, però es poden generar fàcilment executant les funcions corresponents). Per altra banda, en la carpeta src es troba l'arxiu \texttt{main.ipynb} amb tot el codi Python utilitzat en aquest projecte. L'estructura d'aquest arxiu Python consisteix principalment en moltes cel·les on es declaren funcions per cada tasca en concret. Cap al final del document Python comença a haver les primeres crides a funcions que criden amb l'ordre correcte a la resta de funcions. Totes les execucions d'experiments passen per la funció \texttt{run\_experiment()}, ja que és la que s'encarrega de que es cridin ordenadament a totes les funcions i amb els paràmetres pertinents.