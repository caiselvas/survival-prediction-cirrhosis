\section{Introducció}
La base de dades utilitzada en aquest treball és el `Cirrhosis Patient Survival Prediction Dataset', disponible a través de la Universitat de Califòrnia a Irvine (UCI) \cite{misc_cirrhosis_patient_survival_prediction_878}. Aquest conjunt de dades conté informació rellevant sobre pacients amb cirrosi hepàtica, incloent 17 característiques clíniques per cada pacient. L'objectiu principal d'aquesta base de dades és permetre la predicció de la supervivència dels pacients amb aquesta condició.

Les característiques clíniques incloses en el conjunt de dades proporcionen una àmplia gamma de dades sobre els pacients, inclòs l'estat de supervivència, que es tracta com a variable objectiu en aquest estudi.

El treball amb aquesta base de dades comporta diverses etapes importants, com ara l'anàlisi estadístic de les variables de manera independent, l'estudi del balanceig de classes, la identificació i gestió de valors perduts (missings) i atípics (outliers), la recodificació de variables si cal, i el particionat del conjunt de dades en subconjunts per a l'entrenament i la validació dels models de predicció de la variable objectiu.

A través d'aquest procés, es buscarà no només desenvolupar habilitats tècniques en manipulació de dades, programació i modelatge estadístic, sinó també cultivar una comprensió més profunda de com l'intel·ligència artificial pot ser aplicada en contextos reals i significatius, com és el cas de la salut i la medicina.

