\section{Introducció}

\subsection{Base de dades}
\cite{misc_cirrhosis_patient_survival_prediction_878}

La base de dades utilitzada en aquest treball és el "Cirrhosis Patient Survival Prediction Dataset", disponible a través de la Universitat de Califòrnia a Irvine (UCI). Aquest conjunt de dades conté informació rellevant sobre pacients amb cirrosi hepàtica, incloent 17 característiques clíniques per cada pacient. L'objectiu principal d'aquesta base de dades és permetre la predicció de la supervivència dels pacients amb aquesta condició.

Les característiques clíniques incloses en el conjunt de dades proporcionen una àmplia gamma de dades sobre els pacients, inclòs l'estat de supervivència, que es tracta com a variable objectiu en aquest estudi. Aquestes dades proporcionen una oportunitat única per als estudiants de la Universitat Politècnica de Catalunya de desenvolupar i provar models predictius utilitzant tècniques d'intel·ligència artificial.

El treball amb aquesta base de dades comporta diverses etapes importants, com ara l'anàlisi estadístic de les variables de manera independent, l'estudi del balanceig de classes, la identificació i gestió de valors perduts (missings) i atípics (outliers), la recodificació de variables si cal, i el particionat del conjunt de dades en subconjunts per a l'entrenament i la validació dels models.

A més de les tasques tècniques de preprocessament de dades, aquesta base de dades permet als estudiants explorar la interacció entre diverses variables clíniques i la seva relació amb la supervivència de pacients amb cirrosi, una aplicació pràctica valuosa en el camp de la salut. Aquesta exploració no només millora les habilitats tècniques dels estudiants en l'anàlisi de dades i el modelatge predictiu, sinó que també els proporciona una comprensió més profunda de com l'intel·ligència artificial pot ser aplicada en contextos de la vida real.

\subsection{Descripció del projecte}

En aquest treball, explorem un conjunt de dades fascinant obtingut de la Universitat de Califòrnia a Irvine (UCI), específicament el "Cirrhosis Patient Survival Prediction Dataset". L'objectiu principal d'aquest estudi és aplicar i analitzar diferents tècniques d'intel·ligència artificial per predir la supervivència de pacients amb cirrosi. Aquesta tasca, malgrat que no busca avançar en la recerca mèdica sobre la cirrosi, serveix com un exercici valuós per als estudiants de grau en Intel·ligència Artificial, com és el cas del present treball realitzat a la Universitat Politècnica de Catalunya.

El treball se centra en diverses fases clau: la neteja i preparació de les dades, l'exploració i anàlisi estadística, la selecció i aplicació de models predictius, i finalment, l'avaluació del rendiment d'aquests models. A través d'aquest procés, es busca no només desenvolupar habilitats tècniques en manipulació de dades, programació i modelatge estadístic, sinó també cultivar una comprensió més profunda de com l'intel·ligència artificial pot ser aplicada en contextos reals i significatius, com és el cas de la salut i la medicina.

En la primera part del treball, es realitza una neteja i normalització dels conjunts de dades, identificant i tractant valors perduts, així com estandarditzant les mesures per a una anàlisi coherent. Aquesta fase és crítica, ja que la qualitat de les dades afecta directament la precisió dels models predictius.

Seguidament, es procedeix amb una exploració detallada de les dades, utilitzant tècniques d'anàlisi exploratòria de dades (EDA) per obtenir una comprensió més profunda de les característiques i tendències presents. Això inclou l'anàlisi de correlacions, distribucions de variables i altres estadístiques descriptives.

La part central del treball se centra en la selecció i aplicació de diversos models d'aprenentatge automàtic. Es consideren tècniques com ara regressió logística, màquines de suport vectorial (SVM), xarxes neuronals, i possiblement models més avançats com ara els arbres de decisió i els boscos aleatoris. Per a cada model, es realitza una avaluació rigorosa utilitzant mètriques com l'exactitud, la precisió, la sensibilitat, i l'especificitat.

Finalment, es presenten les conclusions extretes de l'anàlisi, destacant els models més efectius i discutint les possibles implicacions dels resultats. A més, es reflexiona sobre les limitacions de l'estudi i es proposen direccions per a futures investigacions.

Aquest treball no només demostra l'aplicació pràctica de l'intel·ligència artificial en el camp de la salut, sinó que també proporciona als estudiants de la UPC una experiència valuosa en l'àmbit de l'anàlisi de dades i modelatge predictiu.